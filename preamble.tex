\usepackage[utf8]{inputenc}
\usepackage[T1]{fontenc}

\usepackage{amsmath}
\usepackage{amsfonts}
\usepackage{amsthm}
\usepackage{mathtools}
\usepackage{amssymb}
\usepackage{enumitem}

\usepackage[style=alphabetic-verb]{biblatex}
\addbibresource{references.bib}
%mathtools extensions
\DeclarePairedDelimiter{\abs}{\lvert}{\rvert}
\DeclarePairedDelimiter{\norm}{\lVert}{\rVert}
\DeclarePairedDelimiter{\iprod}{\langle}{\rangle}

\usepackage{microtype}

\usepackage{tikz-cd,tikz}
    \usetikzlibrary{calc}

\usepackage{subfiles}
\usepackage{titlesec}

\usepackage{comment} 

\usepackage{titlesec}
\titleformat{\chapter}[hang]
{\scshape\fillast}
{\thechapter .}
{0.5em}
{}

\usepackage{fancyhdr}
\fancyhf{}
\renewcommand{\chaptermark}[1]{\markboth{#1}{}}
\fancyhead[C]{\scshape Chapter \thechapter: \leftmark }
\renewcommand{\headrulewidth}{0.3pt}
\renewcommand{\footrulewidth}{\iffootnote{0pt}{0.3pt}}

\fancyfoot[C]{\thepage}

\titleformat{\part}[display]
{\centering\scshape\huge}
{Part \thepart}
{0.5em}
{}

\let\oldprintbibliography=\printbibliography

\def\printbibliography{\clearpage
\pagestyle{plain}
\oldprintbibliography
\clearpage
\pagestyle{fancy}}

%AMS-style theorem environments:

%all of these follow the same counter, which is relative to each chapter:
\theoremstyle{remark}
\newtheorem{theorem}{Theorem}[chapter]
\newtheorem{proposition}[theorem]{Proposition}
\newtheorem{preposition}[theorem]{Preposition}
\newtheorem{lemma}[theorem]{Lemma}
\newtheorem{corollary}[theorem]{Corollary}
\newtheorem{example}[theorem]{Example}
\newtheorem{remark}[theorem]{Remark}
\newtheorem{condition}[theorem]{Condition}

%numberless env's:
\newtheorem*{definition}{Definition}
\newtheorem*{examples}{Example}
\newtheorem*{exercise}{Exercise}
\newtheorem*{notes}{Notes}
\newtheorem*{note}{Note}
\newtheorem*{hint}{Hint}
\newtheorem*{theorem*}{Theorem}
\newtheorem*{remark*}{Remark}
\newtheorem*{explanation}{Explanation}
\newtheorem*{warning}{Warnings}

%examples with their own counters
\newtheorem{example_different}{Example}

%solution for custom numbering in theorems and lemmata:
\newtheorem{innercustomgeneric}{\customgenericname}
\providecommand{\customgenericname}{}
\newcommand{\newcustomtheorem}[2]{%
  \newenvironment{#1}[1]
  {%
   \renewcommand\customgenericname{#2}%
   \renewcommand\theinnercustomgeneric{##1}%
   \innercustomgeneric
  }
  {\endinnercustomgeneric}
}
\newcustomtheorem{customthm}{Theorem}
\newcustomtheorem{customlemma}{Lemma}
\newcustomtheorem{customcor}{Corollary}
\newcustomtheorem{customprop}{Proposition}
\newcustomtheorem{customexample}{Examples}
\newcustomtheorem{customremark}{Remark}

%%%%%

%%% Used for downsizing tikzcd diagrams, don't remove. Extremely important.
\usepackage{adjustbox}

\usepackage{setspace}
\linespread{1.25}%for spacing

\usepackage[b5paper]{geometry}%page size

\usepackage{xcolor}
\newcommand{\plscite}[1]{{\color{purple} #1}} %Changed from green to purple to stop the eye massacre


%fix for equals arrows
%copy & paste
%\ar[-,double line with arrow={-,-}]{r}
\usetikzlibrary{decorations.markings}
\tikzset{double line with arrow/.style args={#1,#2}{decorate,decoration={markings,%
mark=at position 0 with {\coordinate (ta-base-1) at (0,1pt);
\coordinate (ta-base-2) at (0,-1pt);},
mark=at position 1 with {\draw[#1] (ta-base-1) -- (0,1pt);
\draw[#2] (ta-base-2) -- (0,-1pt);
}}}}


\makeindex


\newcommand{\lal}[1]{\overset{#1}{\longleftarrow}} %same thing above but left instead of right
\newcommand{\lra}{\longrightarrow} %just a shorthand for longrightarrow

\newcommand{\te}[1]{\mathrm{#1}} %for functions with like, text names. for example do \te{Cyl}(X)

\makeatletter

\def\@@varrightarrow#1#2#3{\begingroup%
\setbox0=\hbox{$#1\xrightarrow[#3]{#2}$}%
\setbox1=\hbox{$#1\longrightarrow$}%
\ifdim\wd0<\wd1 \mathrel{\mathop{\longrightarrow}\limits^{#2}_{#3}}
\else \xrightarrow[#3]{#2} \fi\endgroup}

\def\@varrightarrow#1#2{\@@varrightarrow#1#2}

% \varrightarrow[under material]{over material} puts material over and under a right arrow.
% It automatically prints the longer option between an xrightarrow and a longrightarrow
\newcommand\varrightarrow[2][]{\mathpalette\@varrightarrow{{#2}{#1}}}

% Because y'all like shortcuts so much
\let\vra=\varrightarrow

\newcommand{\lar}[1]{\vra{#1}} %shorthand for longrightarrow with something set above it

\makeatother

\newcommand{\nth}{^{\text{th}}} %numbers


% BLACKBOARD BOLD
\let\mbb=\mathbb
\newcommand{\bbZ}{\mathbb{Z}}
\newcommand{\bbQ}{\mathbb{Q}}
\newcommand{\bbR}{\mathbb{R}}
\newcommand{\bbC}{\mathbb{C}}
%TeXit's blackboard bold
\newcommand{\bZ}{\mathbb{Z}}
\newcommand{\bQ}{\mathbb{Q}}
\newcommand{\bR}{\mathbb{R}}
\newcommand{\bC}{\mathbb{C}}

%% Fixing greek-vargreek

% gobbles a single token
\def\gobble#1{}

% \prependmacro{<text>}\cs -> \<text>cs
\def\prependmacro#1#2{\csname #1\expandafter\gobble\string#2\endcsname}

% \restoremacro\cs1\cs2 -> \let\oldcs1=\cs1
%                          \let\cs1=\cs2
\def\restoremacro#1#2{%
    \expandafter\expandafter\expandafter\let\prependmacro{old}{#1}=#1 %
    \let#1=#2 %
}

\restoremacro\phi\varphi

% some operators
\newcommand{\Hom}{\te{Hom}}
\newcommand{\Int}{\te{Int}}
\newcommand{\Ext}{\te{Ext}}
\DeclareMathOperator{\Tor}{\te{Tor}}
\DeclareMathOperator{\Ker}{\te{Ker}}

\DeclareMathOperator{\MU}{MU}
\DeclareMathOperator{\BU}{BU}
\newcommand{\cf}{\mathrm{cf}}

\newcommand{\limi}{\mathop{\mathrm{lim}^1}}
\newcommand{\limo}{\mathop{\mathrm{lim}^0}}

\usepackage[bottom]{footmisc}

\renewcommand{\tilde}[1]{\widetilde{#1}}

% hyperref should be loaded last
\usepackage[
	bookmarks = true, 
	colorlinks = true, 
	linkcolor = blue, 
	citecolor = red
]{hyperref}

\makeatletter
\@addtoreset{chapter}{part}
\makeatother
